\documentclass[11pt, a4paper,titlepage]{article}
\usepackage[hidelinks]{hyperref}
\author{Edoardo Giacomello \and Mattia Fontana}
\title{SE2 Requirement Analysis and Specification Document} %Definisce il titolo
\newcommand{\productname}{MyTaxiService }
\begin{document}
%Genera il titolo
\maketitle
%Inserisce tabella dei contenuti
\tableofcontents
\newpage
\section{Introduction}
\subsection{Purpose} This document represents the Requirement Analysis and Specification Document (RASD). The main goal of this document is to completely describe the system in terms of functional and non-functional requirements, analyse the real need of the customer to modelling the system, show the constraints and the limit of the software and simulate the typical use cases that will occur after the development. This document is intended to all developer and programmer who have to implement the requirements, to system analyst who want to integrate other system with this one, and could be used as a contractual basis between the customer and the developer.
\subsection{Scope}
The scope of \productname is to manage a taxi service in a more efficient way, along with an emprovement in the interaction between the customers and the service providers. 
\newline In particular, the following goals have be highlighted by the stakeholders:
\subsubsection{Goals}
	\begin{enumerate}
	\item \label{itm:Goal_PassengerInterface} Providing the passenger a \textit{\textbf{\hyperref[itm:Desc_SimplifiedAccess]{simplified access}}} for making real-time request for the taxi service, according to appropriate \textit{\textbf{\hyperref[itm:Desc_UsabilityMetric]{usability metrics}}}.
	\item \label{itm:Goal_Reservation} Providing the passenger the possibility to reserve a taxi for a later moment.
	\item \label{itm:Goal_Confirmation} Providing the passenger a clear confirmation of the taxi ride he’s requesting for.
	\item \label{itm:Goal_WaitingTime} Providing the passenger an estimation of the waiting time for a taxi ride.
	\item \label{itm:Goal_TaxiInterface} Providing the taxi drivers a \textit{\textbf{\hyperref[itm:Desc_SimplifiedAccess]{simplified access}}} for receiving and handling transportation requests from the passengers.
	\item \label{itm:Goal_FairManagement} Guaranteeing a fair management of the taxi queue, in term of minimizing the passenger \textit{\textbf{\hyperref[itm:Desc_WaitingTime]{waiting time}}}.
	\item \label{itm:Goal_DeveloperInterface} Providing \textit{\textbf{\hyperref[itm:Actor_SysDevs]{system developers}}}  a programmatic interface to further extend the system.
		\end{enumerate}
\subsubsection{Actors and Stakeholders}
\begin{description}
	\item[Customer] \label{itm:Actor_Customer} The person(s) who requested the developing of the system. Also referred as Government in the assignment text.
	\item[Passenger] \label{itm:Actor_Passenger} The person who avails of the taxi service, in particular those who make a request for a Taxi ride.
	\item[Taxi Driver] \label{itm:Actor_TaxiDriver} The employee which is in charge to meet the passenger and take him to the desired location
	\item[User]\label{itm:Actor_User} General term for describing whoever interacts with the system interfaces. It can be a passenger, a taxi driver, a system developer, a system administrator, etc.
	\item[System Developers] \label{itm:Actor_SysDevs} The persons which are qualified and in charge to extend the present system with additional features or services
	\item[System Administrator] \label{itm:Actor_SysAdmin}: The persons in charge to manage the relationship between the transportation service and the system. For example a system administrator could register new taxi drivers’ profiles, manage the passenger profiles or manage data backups.
\end{description}

\subsection{Definitions, Acronyms, Abbreviations}
\subsubsection{Definitions}
	\begin{description}
		\item[Simplified Access] \label{itm:Desc_SimplifiedAccess} Human-to-Machine interaction that satisfy some usability metrics
		\item[Usability Metric] \label{itm:Desc_UsabilityMetric} Precise measurable requirements that an interface has to satisfy in order to provide an efficient and easy experience to the user.
		\item[Waiting time] \label{itm:Desc_WaitingTime} Time in minutes between the submission of a request and the arrival of a taxi.
	\end{description}
\subsubsection{Acronyms}
\subsubsection{Abbreviations}
\subsection{Reference Documents}
	\begin{itemize}
		\item Specification Document: Assignments 1 and 2 (RASD and DD).pdf.
		\item Specification Document:Project Description And Rules.pdf
		\item IEEE Std 830-1998 IEEE Recommended Practice for Software Requirements Specifications.
		
	\end{itemize}
\subsection{Document Overview}
The following part of this document will focus on a deeper analysis of the goals highlighted by the stakeholders in addition to some assumptions that must hold in order to keep the system properties valid. The last part of the document will exhibit the software requirements that resulted from the goals examination along with some scenario and use cases that will offer the stakeholders a better comprehension of the final system.   
\section{Overall description}
This section will cover the overall description of the product. In particular it will put this product into the perspective of other products or systems. 
\subsection{Product perspective}
The \productname system will partially replace any former traditional system that is based on phone calls with a new interactive system based on mobile and web interfaces. \newline
The decision of completely dismissing the former phone call system in favor to the new one will be left to the customer, since \productname requires the passengers to access a web interface or a mobile application, which is still generally not accessible by some users.
\newline
The project will therefore consists in releasing a web application and a mobile app which are not integrated with any other existing taxi management service.
The system is created to improve the possibility to connect with the taxi service for requesting a taxi, that is based on phone calls.
With this system, the taxi service can provide a fast and innovative system that allow to manage the user request rapidly, without any loss of time.
Both the mobile application and the web interface are connected to the central system and allow the user to know in real time the waiting time, the taxi occupation and their location, and allows to make a taxi request receiving a rapid response.
The application will furthermore provide a programmatic interface for the integration with future systems and extensions.
\subsubsection{System Interfaces}
	\begin{enumerate}
		\item \productname mobile application will run on tablets and smartphones
		\item \productname web application will run on every terminal which is provided an internet connection and a web browser
	\end{enumerate}
\subsubsection{User Interfaces}
Both the web and mobile applications will have a similar user interface, rearranged to fit the device screen.

\textbf{Application for passengers}
It has to provide a login page that allows users to be recognized by the system, along with a registration form that let the user to sign up to the service and a link for resetting the user password.
In the home page after the login, it is possible to see the map of the zone in which the user is and the taxis that are nearby, along with a menu for accessing the other functionalities.
%TODO: Finish this
\subsubsection{Hardware Interfaces}
The \productname mobile application will require a smartphone or tablet and will support the most common screen resolutions. \newline
For the correct functionality of \productname mobile and web application (both for passenger and taxi drivers) the device will be required to support the GPS localization and a WiFi or data internet connection. 
\subsubsection{Software Interfaces}
\begin{itemize}
	\item \productname Application Server
		\begin{itemize}
			\item Database Management System (DBMS):
			\begin{itemize}
				\item[Name]: MySQL.
				\item[Version]:  5.6.19
				\item[Source]: http://www.mysql.it/
			\end{itemize}
			\item Application server:
			\begin{itemize}
				\item[Name]:  Jboss application server J2EE open source.
				\item[Source]: http://www.jboss.org/
			\end{itemize}
			\item Operative Systems:
			\begin{itemize}
				\item: Ubuntu/Debian Linux
				\item: Windows Server, Windows 7, Windows 8, Windows 10
			\end{itemize}
		\end{itemize}
		
	\item \productname Application Server
	\begin{itemize}
		\item Mobile Application
		\begin{itemize}
			\item[OS]: Android 4.4.2 or higher, iOS, Windows Mobile.
		\end{itemize}
	\end{itemize}
	\begin{itemize}
		\item Web Application
		\begin{itemize}
			\item[Web Browsers]: Chrome, Firefox, Safari, Opera, Internet Explorer
			\item[Other]: JRE 1.7 or higher
		\end{itemize}
	\end{itemize}	
		
\end{itemize}
\subsubsection{Communications Interfaces}
	The communication between the server and mobile application will make use of the following ports:
	\newline \newline
	\begin{tabular}{c c c}
		\textbf{Protocol} & \textbf{Application} & \textbf{Port} \\
		\hline
		TCP & HTTP & 80 \\
		\hline
		TCP & HTTPS & 443 \\
		\hline
		TCP & DBMS & 3306 \\
		\hline
	\end{tabular}
	\newline
	The default communication protocol will be HTTPS.
	
\subsubsection{Memory Constraints}
The suggested memory requirements are: \newline
\begin{itemize}
	\item Application Server
	\begin{itemize}
		\item[Main Memory]: 4GB
		\item[Secondary Memory]: 2GB
	\end{itemize}
	\item Mobile Application
	\begin{itemize}
			\item[Main Memory]: 500Mb
			\item[Secondary Memory]: 100Mb
	\end{itemize}
\end{itemize}
\subsubsection{Operations}
	In this section will be described what operations each kind of user can do by interacting with the system interface: \newline
	\begin{enumerate}
		\item Passengers (Web and Mobile Application)
			\begin{description}
				\item[Registration] The user can create a new account filling in his personal informations
				\item[Login] The user can access the main page by inserting his email and password.
				\item[Map] The user can visualize the map of the zone he is in.
				\item[Request] The user can request a taxi immediately, or adding to the queue according to the taxi availability
				\item[Reservation] The user can make a reservation of a taxi ride for a later moment
				\item[Pending Reservation] The user can visualize data such the estimated taxi arrival time about the reservation he made and cancel a reservation.
				\item[Information] The user can access some information about the system usage and rules
				\item[Profile] The user can access its profile personal data
				\item[Logout] The user can logout from the system
			\end{description}
			
		\item Taxi Drivers (Mobile Application Only)
			\begin{description}
				\item[Login] The user can access the main page by inserting his taxi code and password. This will state the availability of accepting requests.
				\item[Map] The user can see a map of the nearby area. When a request arrives the map will show the meeting point
				\item[Request Incoming] The user can accept or refuse an incoming request
				\item[Active Ride] The user can complete a ride when the passenger is at his desired location, or release the ride to another taxi driver if some unexpected event occurs.
				\item[Logout] The user can logout from the system and state his unavailability of taking requests.
			\end{description}
	
		\item System Administrators (Web Interface Only)  
			\begin{description}
				\item[Login] The user can login and access the administration page
				\item[Taxi Management] The user can acess the list of all active and unactive taxi accounts. He can Add and Remove taxi driver accounts or change their passwords.
				\item [User Management] The user can access user profiles to provide support to end users and reset their passwords.
				\item[Logs] The user can view and download the logs associated to each accepted/refused request, all ride data such passenger account, departure/arrival date and location, and the taxi driver who made the ride.
				\item[Backup/Restore] The user can make security backups of account data and restore them.
			\end{description}
		
		\item System Operations
		\begin{description}
			\item[Store user information] Creates or updates a passenger profile
			\item[Store taxi information] Creates or updates a taxi profile
			\item[Update Map] Update the map every 30 seconds with each taxi location
			\item[Registration Confirmation] The system sends a confirmation email to the user upon registration
			\item[Request Confirmation] The system sends a confirmation email about the acceptance of a taxi request
			\item[Reservation Confirmation]  The system sends a confirmation email about the acceptance of a taxi reservation
			\item[Reservation Reminder] The system sends a remainder email to the user who reserved a taxi 2 hour before the meeting time
			\item[Time estimation] The system estimate the time needed to move by car from a location A to a location B. It can make use of external map services for this purpose.
		\end{description}
		
		\item[System Support Operation]
			\begin{description}
				\item[Backup Database] The system make a backup of the database and logs
				\item[Restore Database] The system make a restore of a previous backup of the database
				\item
			\end{description}
			
	\end{enumerate}
\subsection{Product Functions}
\subsection{User charateristics}
\subsection{Constraints}
\subsubsection{Regulatory Policies}
The system shall comply with the following regulatory policies:
\begin{itemize}
	\item CODICE IN MATERIA DI PROTEZIONE DEI DATI PERSONALI "Decreto legislativo 30 giugno 2003, n. 196"
\end{itemize}
\subsubsection{Hardware Limitations}
\subsubsection{Interfaces to other applications}
\subsubsection{Reliability Requirements}
\subsubsection{Criticality of the application}
%Spiegare qui le usability metrics utilizzate e il fatto che non devono esserci sprechi di risorse, come taxi inutilizzati o clienti non serviti
\subsubsection{Safety and Security Consideration}
\productname should comply to the serurity statements listed above: \newline
\begin{itemize}
	\item Only logged users can request a taxi ride. This is for preventing any malicious behaviour in which unknown users can make taxi request without making use of the transportation service.
	\item At the time of registration, a user must read and accept any privacy policy about the user data management
	\item The system must undertake to not divulge the user data to third parties, in addition the user data will be used only for the supply of the service
	\item For the reason stated before, the user MUST supply:
		\begin{itemize}
			\item Name
			\item Surname
			\item e-mail address
			\item Phone number or Tax Code
		\end{itemize}
	\item The system should therefore warn the user about the regulation policy about false impersonation.
	\item The system will hash the user password before storing it into the database, preventing unauthorized access in the case of data leak.
	\item The system will ask the user the email address or the phone number in order to reset his password. The new password will be sent on the user e-mail address.
\end{itemize}
\subsection{Assumptions and Dependancies}
\subsubsection{Assumptions}
	\begin{enumerate}
		\item \label{itm: Assumption_Zones} The city is already divided in Taxi Zones of approximately 2x2 km each. 
		\item \label{itm: Assumption_MobileProvisioning} Taxis or Taxi Drivers are assumed to be in possession of a mobile device that  1) have GPS functionalities and 2) is capable of establishing an internet connection with the facility in which the system is hosted. In the case one of these assumption does not hold, the Taxi driver must contact his referent in order to solve the issue.
		\item \label{itm: Assumption_NumberOfTaxis} The number of taxi in service are at least equal to the number of zones.
		\textbf{Rationale}: In this way, in the case that a zone queue is empty, an incoming request is guaranteed to be served by at least one taxi in an estimable time by propagating the request to adjacent zone and forwarding the request to the first taxi that is either available or have the scheduled arrival location in that zone.
		\item \label{itm: Assumption_FormerSystems} The service is currently running a traditional system based on phone calls	
		\item \label{itm: Assumption_Payment} The payment will occur directly to the taxi driver by cash or credit card and it will be managed by the transportation company
		
	\end{enumerate}

\section{Specific Requirements}

\appendix
\end{document}
