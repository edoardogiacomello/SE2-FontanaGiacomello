\documentclass[11pt, a4paper,titlepage]{article}
\usepackage[hidelinks]{hyperref}
\usepackage{graphicx}
\usepackage{tabularx}
\usepackage{booktabs}
\usepackage{multirow}
\usepackage[table]{xcolor}
\usepackage[normalem]{ulem}
\useunder{\uline}{\ul}{}
\graphicspath{{img/}}
\author{Edoardo Giacomello \and Mattia Fontana}
\title{SE2 Project Plan Document} 
%Definisce il titolo
\newcommand{\productname}{MyTaxiService}
\newcommand{\image}[1]{
	\begin{center}
		\noindent \includegraphics[width=\linewidth]{#1}
	\end{center}
}
\newcommand{\link}[2]{\underline{\textbf{\hyperref[#1]{#2}}}}
\newcommand{\linkitm}[1]{\underline{\textbf{\ref{#1}}}}


\begin{document}
	%Genera il titolo
	\maketitle
	%Inserisce tabella dei contenuti
	\tableofcontents
	\newpage
	\section{Introduction}
	\subsection{Revision History}
	
	\subsection{Purpose and Scope}
	
	\subsubsection{Purpose}
	The purpose of this document is to provide an estimation of the size, the costs and the possible risks in the development context for the \productname System. \newline
	\subsubsection{Scope}
	The scope of this document is the planning and allocation of resources designated at the \productname software development. This document represent a starting point in cost and effort estimation.  
	\subsection{List of Definitions and Abbreviations}
	\begin{description}
		\item[FP - Function Points]: Function points measure a software project by quantifying the information processing functionality associated with major external data or control input, output, or file types. 
		\item[RET - Record Element Type]: A RET is user recognizable sub group of data elements within an ILF or an EIF.
		\item[DET - Data Element Type] A DET is a unique user recognizable, non-recursive (non-repetitive) field. A DET is information that is dynamic and not static. A dynamic field is read from a file or created from DET’s contained in a FTR. Additionally, a DET can invoke transactions or can be additional information regarding transactions. If a DET is recursive then only the first occurrence of the DET is considered not every occurrence.
		\item[EI - External Input]: Every unique user data or user control input type that enters the external boundary of the software system being measured
		\item[EO - External Output]: Every unique user data or control output type that leaves the external
		boundary of the software system being measured. 
		\item[ILF - Internal Logical File]: Every major logical group of user data or control information that is generated, used or maintained by the software system
		\item[EIF - External Interface File]: Files that are passed or shared between different software systems.
		\item[EQ - External Inquiry] Every unique input-output combination, where input causes and generates an immediate output.
	\end{description}
	\subsection{List of Reference Documents}
	\begin{itemize}
		\item Assigment 1: Project Description
		\item \productname Requirement and Specification Analysis Document
		\item \productname Design Document
	\end{itemize}
	\newpage
	\section{Project Size and Cost Estimation}
		\subsection{Project Size: Function Points}
		\subsubsection{Complexity Weights}
		In the evaluation of the complexity weights we referred to this table: \newline
		\image{fp_weights.png}
		\newpage
		\subsubsection{ILF: Internal Logical Files}
		The data model resides in the database, which is typically counted as a single ILF which is structured as follow:
		\newline
		\begin{tabularx}{\textwidth}{|X|c|c|l|}
			\hline
			\textbf{Name} &	\textbf{RET} & \textbf{DET} & \textbf{Complexity} \\
			\hline
			Zone & 1 & 6 & Low \\
			Location & 1 & 3 & Low \\
			Taxi & 1 & 5 & Low \\
			Requests and Reservations & 1 & 8 & Low \\
			Account & 1 & 8 & Low \\
			Logs & 1 & $\le$ 4 & Low \\
			\hline
		\end{tabularx}
		
		\subsubsection{EIFs (External Interface Files)}
		The system does not rely on files that resides on external systems
		\subsubsection{EIs (External Inputs)}
				
				\begin{tabularx}{\textwidth}{|X|X|X|l|}
					\hline
					\textbf{Name} &	\textbf{RET} & \textbf{DET} & \textbf{Complexity} \\
					\hline
					Login / Logout / Registration & 1: Account & $\le$8 & Low \\
					\hline
					TaxiProbe & 1: Location & 3 & Low \\
					\hline
					TaxiReservation input and processing & 2 \newline  Account, Reservation, Zones, Taxi  & 16 & High \\
					\hline
					DriverResponse & 1 & 3 \newline (ReqId + TaxiId + Accepted) & Low \\
					\hline
					DriverNotification & 1 & 3 \newline (ReqId + TaxiId + Completed) & Low \\
					\hline
					LocationUpdate & 1: Location & 3 & Low \\
					\hline
					BackupDatabase & n.a. &   & Avg \\
					\hline
					RestoreDatabase & n.a &   & Avg \\
					\hline
				\end{tabularx}
				\newpage
		\subsubsection{EIQs (External Inquiries)}
		\begin{tabularx}{\textwidth}{|X|c|c|l|}
			\hline
			\textbf{Name} &	\textbf{RET} & \textbf{DET} & \textbf{Complexity} \\
			\hline
			ShowProfile & 1 & 8 & Low \\
			ShowAccounts & 1 & 8 & Low \\
			ShowTaxiList & 2 & 16 & Avg \\
			ShowLogs  & n.a. &   & Avg \\
			\hline
		\end{tabularx}
		\subsubsection{EOs (External Outputs)}
		\begin{tabularx}{\textwidth}{|X|c|c|l|}
			\hline
			\textbf{Name} &	\textbf{RET} & \textbf{DET} & \textbf{Complexity} \\
			\hline
			TaxiProbeResponse &	1 &	5 \newline (Taxi + Location) & Low \\
			Confirmation & 1\newline Reservation & 8 & Low \\
			Notification & 1\newline Reservation & 8 & Low \\
			DriverRequest & 1\newline Reservation & 8 & Low \\
			\hline
		\end{tabularx}
		\subsubsection{UFPs (Un-adjusted Function-Points)}
		\begin{tabularx}{\textwidth}{|X|c|c|c|l|}
			\hline
			\textbf{Name} &	\textbf{Low} & \textbf{Avg} & \textbf{High} & \textbf{Total}\\
			\hline
			ILF &	6*7 &	0 &	0 &	42 \\
			EIFs & 0  &	0 &	0 &	0 \\
			EIs & 5*3 & 2 *4 & 1*6 & 29 \\
			EIQs & 2*3 & 2*4 & 0 & 14 \\
			EOs & 4*4 & 0 & 0 & 16 \\
			\hline
			UFP & 79 & 16 & 6 & \textbf{101} \\			
			\hline
		\end{tabularx}
		\newpage
		\subsection{Effort and Cost Estimation: COCOMO}
	\section{Tasks and Schedule}
	\section{Resources Allocation}
	\section{Risks and Management}








	\section{Work Hours}
		\begin{itemize}
			\item \textbf{Edoardo Giacomello}: 
			\item \textbf{Mattia Fontana}: 
		\end{itemize}	
\end{document}