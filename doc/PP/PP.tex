\documentclass[11pt, a4paper,titlepage]{article}
\usepackage[hidelinks]{hyperref}
\usepackage{graphicx}
\usepackage{tabularx}
\usepackage{booktabs}
\usepackage{multirow}
\usepackage[table]{xcolor}
\usepackage[normalem]{ulem}
\useunder{\uline}{\ul}{}
\graphicspath{{img/}}
\author{Edoardo Giacomello \and Mattia Fontana}
\title{SE2 Project Plan Document} 
%Definisce il titolo
\newcommand{\productname}{MyTaxiService}
\newcommand{\image}[1]{
	\begin{center}
		\noindent \includegraphics[width=\linewidth]{#1}
	\end{center}
}
\newcommand{\link}[2]{\underline{\textbf{\hyperref[#1]{#2}}}}
\newcommand{\linkitm}[1]{\underline{\textbf{\ref{#1}}}}


\begin{document}
	%Genera il titolo
	\maketitle
	%Inserisce tabella dei contenuti
	\tableofcontents
	\newpage
	\section{Introduction}
	\subsection{Revision History}
	
	\subsection{Purpose and Scope}
	
	\subsubsection{Purpose}
	The purpose of this document is to provide an estimation of the size, the costs and the possible risks in the development context for the \productname System. \newline
	\subsubsection{Scope}
	The scope of this document is the planning and allocation of resources designated at the \productname software development. This document represent a starting point in cost and effort estimation.  
	\subsection{List of Definitions and Abbreviations}
	\begin{description}
		\item[FP - Function Points]: Function points measure a software project by quantifying the information processing functionality associated with major external data or control input, output, or file types. 
		\item[RET - Record Element Type]: A RET is user recognizable sub group of data elements within an ILF or an EIF.
		\item[DET - Data Element Type] A DET is a unique user recognizable, non-recursive (non-repetitive) field. A DET is information that is dynamic and not static. A dynamic field is read from a file or created from DET’s contained in a FTR. Additionally, a DET can invoke transactions or can be additional information regarding transactions. If a DET is recursive then only the first occurrence of the DET is considered not every occurrence.
		\item[EI - External Input]: Every unique user data or user control input type that enters the external boundary of the software system being measured
		\item[EO - External Output]: Every unique user data or control output type that leaves the external
		boundary of the software system being measured. 
		\item[ILF - Internal Logical File]: Every major logical group of user data or control information that is generated, used or maintained by the software system
		\item[EIF - External Interface File]: Files that are passed or shared between different software systems.
		\item[EQ - External Inquiry] Every unique input-output combination, where input causes and generates an immediate output.
	\end{description}
	\subsection{List of Reference Documents}
	\begin{itemize}
		\item Assigment 1: Project Description
		\item \productname Requirement and Specification Analysis Document
		\item \productname Design Document
	\end{itemize}
	\newpage
	\section{Project Size and Cost Estimation}
		\subsection{Project Size: Function Points}
		\subsubsection{Complexity Weights}
		In the evaluation of the complexity weights we referred to this table: \newline
		\image{fp_weights.png}
		\newpage
		\subsubsection{ILF: Internal Logical Files}
		The data model resides in the database, which is typically counted as a single ILF which is structured as follow:
		\newline
		\begin{tabularx}{\textwidth}{|X|c|c|l|}
			\hline
			\textbf{Name} &	\textbf{RET} & \textbf{DET} & \textbf{Complexity} \\
			\hline
			Zone & 1 & 6 & Low \\
			Location & 1 & 3 & Low \\
			Taxi & 1 & 5 & Low \\
			Requests and Reservations & 1 & 8 & Low \\
			Account & 1 & 8 & Low \\
			Logs & 1 & $\le$ 4 & Low \\
			\hline
		\end{tabularx}
		
		\subsubsection{EIFs (External Interface Files)}
		The system does not rely on files that resides on external systems
		\subsubsection{EIs (External Inputs)}
				
				\begin{tabularx}{\textwidth}{|X|X|X|l|}
					\hline
					\textbf{Name} &	\textbf{RET} & \textbf{DET} & \textbf{Complexity} \\
					\hline
					Login / Logout / Registration & 1: Account & $\le$8 & Low \\
					\hline
					TaxiProbe & 1: Location & 3 & Low \\
					\hline
					TaxiReservation input and processing & 2 \newline  Account, Reservation, Zones, Taxi  & 16 & High \\
					\hline
					DriverResponse & 1 & 3 \newline (ReqId + TaxiId + Accepted) & Low \\
					\hline
					DriverNotification & 1 & 3 \newline (ReqId + TaxiId + Completed) & Low \\
					\hline
					LocationUpdate & 1: Location & 3 & Low \\
					\hline
					BackupDatabase & n.a. &   & Avg \\
					\hline
					RestoreDatabase & n.a &   & Avg \\
					\hline
				\end{tabularx}
				\newpage
		\subsubsection{EIQs (External Inquiries)}
		\begin{tabularx}{\textwidth}{|X|c|c|l|}
			\hline
			\textbf{Name} &	\textbf{RET} & \textbf{DET} & \textbf{Complexity} \\
			\hline
			ShowProfile & 1 & 8 & Low \\
			ShowAccounts & 1 & 8 & Low \\
			ShowTaxiList & 2 & 16 & Avg \\
			ShowLogs  & n.a. &   & Avg \\
			\hline
		\end{tabularx}
		\subsubsection{EOs (External Outputs)}
		\begin{tabularx}{\textwidth}{|X|c|c|l|}
			\hline
			\textbf{Name} &	\textbf{RET} & \textbf{DET} & \textbf{Complexity} \\
			\hline
			TaxiProbeResponse &	1 &	5 \newline (Taxi + Location) & Low \\
			Confirmation & 1\newline Reservation & 8 & Low \\
			Notification & 1\newline Reservation & 8 & Low \\
			DriverRequest & 1\newline Reservation & 8 & Low \\
			\hline
		\end{tabularx}
		\subsubsection{UFPs (Un-adjusted Function-Points)}
		\begin{tabularx}{\textwidth}{|X|c|c|c|l|}
			\hline
			\textbf{Name} &	\textbf{Low} & \textbf{Avg} & \textbf{High} & \textbf{Total}\\
			\hline
			ILF &	6*7 &	0 &	0 &	42 \\
			EIFs & 0  &	0 &	0 &	0 \\
			EIs & 5*3 & 2 *4 & 1*6 & 29 \\
			EIQs & 2*3 & 2*4 & 0 & 14 \\
			EOs & 4*4 & 0 & 0 & 16 \\
			\hline
			UFP & 79 & 16 & 6 & \textbf{101} \\			
			\hline
		\end{tabularx}
		\newpage
		\subsection{Effort and Cost Estimation: COCOMO}
		All the tables used in this analysis have been taken from COCOMO II, Model Definition
		Manual at:
		\newline
		http://csse.usc.edu/csse/research/COCOMOII/cocomo2000.0/CII\textunderscore modelman2000.0.pdf
		\newline
		Elements of the COCOMO II model:
		\begin{itemize}
			\item Source Lines of Code (SLOC)
			\item Scale Drivers
			\item Cost Drivers
			\item The Effort Equation
			\item The Effort Adjustment Factor
			\item The Schedule Equation
			\item The SCED (Schedule Constraints) Cost Driver
		\end{itemize}
	    \subsubsection{Source Lines of Code (SLOC)}
		
		101 FPs * 53 = 5353  SLOC
		\newline
		Where 53 is found from this table :
		\image{Co_ufp.png}
		\subsubsection{Scale Drivers}
		We use this table :
		\image{Scale_factor.png}
		
		for defining the following parameters: 
		\newline
		\newline
		\begin{tabularx}{\textwidth}{|X|c|c|}
			\hline
			\textbf{Scale Driver} &	\textbf{Factor} & \textbf{Value} \\
			\hline Precedentedness & Low &	4.96  \\
			 Development Flexibility & High & 2.03\\	
			 Architecture / Risk Resolution & Very High & 1.41\\
			 Team Cohesion & Nominal & 3.29\\
			 Process Maturity & High & 3.12\\
			\hline Total : & / & 14.81 \\
			\hline
		\end{tabularx}
		\newline\newline\newline
		Analysis of the results :
		\begin{itemize}
			\item Precedentedness : Low, the team was not expert with business scale projects.
			\item Development Flexibility: High, the project has been structured in a way that facilitates further changes.
			\item Architecture / Risk Resolution: Reflects the extent of risk analysis carried out, most of risk was deleted.
			\item Team Cohesion : Reflects how well the development team know each other and work together, this is Nominal because this is the first project that we do together.
			\item Process Maturity : This was evaluated around the 18 Key Process Area (KPAs) in the SEI Capability
			Model.
		\end{itemize}
	
		\subsubsection{Cost Drivers}
		\begin{tabularx}{\textwidth}{|X|c|c|}
			\hline
			\textbf{Cost Driver} &	\textbf{Factor} & \textbf{Value} \\
			\hline Required Software Reliability & Very Low &	0.82  \\
			 Data Base Size & Low & 0.90\\	
			 Product Complexity & High & 1.17\\
			 Required Reusability & High & 1.07\\
			 Documentation match to life-cycle needs & Nominal & 1.00\\
			 Execution Time Constraint & High & 1.11\\
			 Main Storage Constraint & Nominal & 1.00\\
			 Platform Volatility & Very Low & n/a\\
			 Analyst Capability & High & 0.85\\
			 Programmer Capability & High & 0.88\\
			 Application Experience & Low & 1.10\\
			 Personnel continuity & Very Low & 1.29\\
			 Platform Experience & Low & 1.09\\
			 Language and Tool Experience & Low & 1.09\\
			 Usage of Software Tools & Nominal  & 1.00\\
			 Multisite development & Extra High & 0.80\\
			 Required development schedule & High & 1.00\\			
			\hline Total : & / & 16.17/17=0.95 \\
			\hline
		\end{tabularx}
		\newline\newline
		For define this table we use the information contained in these tables :
		\textbf{Required Software Reliability :} Very Low, some slight inconvenience is allowed if recovered in a short time.
		\image{rely.png}
		\textbf{Data Base Size :} 
		P=5353 SLOC
		D= 640 KB
		D/P=8.364
		\image{data.png}
    	\textbf{Product Complexity:} high according to the new COCOMO II CPLEX rating scale.
		\image{cplx.png}
		\textbf{Required Reusability:}
		\image{ruse.png}
		\textbf{Documentation match to life-cycle needs:}
		\image{docu.png}
		\textbf{Execution Time Constraint:}
		\image{time.png}
		\textbf{Main Storage Constraint:}
		\image{stor.png}
		\textbf{Platform Volatility:}
		\image{pvol.png}
		\textbf{Analyst Capability:}
		\image{acap.png}
		\textbf{Programmer Capability:}
		\image{pcap.png}
		\textbf{Application Experience:}
		\image{apex.png}
		\textbf{Platform Experience:}
		\image{plex.png}
		\textbf{Language and Tool Experience:}
		\image{ltex.png}
		\textbf{Personnel continuity:}
		\image{pcon.png}
		\textbf{Usage of Software Tools:}
		\image{tool.png}
		\textbf{Multisite development:}
		\image{site.png}
		\textbf{Required development schedule:}
		\image{sced.png}
		
		\subsubsection{Effort Equation}
		$ Effort = A * EAF * (KSLOC)^{E} $
	    \newline
		Where:
		\begin{itemize}
			\item 	A = 2.94
			\item EAF = product of all the cost drivers, equal to : 0.95 ;
			\item E = exponent derived from Scale Drivers. Is calculated as: \newline
				B + 0.01 * sum\{i\} SF[i] := B + 0.01 *14.81 = 0.91 + 0.1481 = 1.0581;
				in which B is equal to: 0.91 for COCOMO.2000 .
			\item KSLOC = estimated lines of code using the FP analysis: 5,353
		\end{itemize}
	   $ Effort = 2.94* 0.95* 5,353^{1.0581}  = 16.4816 Person-months $
		
	   \subsubsection{Schedule Estimation}
	   $Duration := 3.67 * Effort^{F}$
	   \newline    
	   Where:
	   	\begin{itemize}
	   		\item F := 0.28 + 0.2 * ( E - B ) = 0.28 + 0.2*(1.0581 - 0.91) = 0.30962
	   	\end{itemize}
	   The estimated project duration is: \newline
	   $ Duration = 3.67 * 16.4816^{0.30962} = 8.739 = 9 Months$
	   \newline
	   The estimation of the team size for this project is:
	   \newline
	   $P = Effort / Duration = 16.4816/ 9 = 1.83 = 2 People $
	   \newline
	   \newline
	   We want to give a more precise estimation adjusting some Scale Driver.
	   To evaluate the COCOMO II and determine the effort required to complete the software project we also used an online tool (http://csse.usc.edu/tools/COCOMOII.php).\newline
	   This is the result of the online tool: 
	   \image{cocomo1.png}
	   \image{cocomo2.png}
		
	\section{Tasks and Schedule}
	
	\subsection{RASD :}
	\begin{itemize}
		\item Identify the project objectives.
		\item Describe the structure and functions of the project.
		\item Define the specific requirements.
		\item Identify the use cases and scenarios.
		\item Representing UML diagrams.
		\item Representing the model in Alloy.
		\item Latex document.
	\end{itemize}

	\subsection{DD :}
	\begin{itemize}
		\item Describe the architecture of the system.
		\item Describe component view and deployment view.
		\item Describe runtime view and component interfaces.
		\item Describe algorithm design.
		\item Define user interface design.
		\item Latex document.
	\end{itemize}
	
	\subsection{Planning:}
	\begin{itemize}
		\item Define Function points.
		\item Define effort and cost estimation : Cocomo.
		\item Describe tasks for the project and the schedule.
		\item Define the risks.
		\item Latex document.
	\end{itemize}
	
	\subsection{Development}
	\begin{itemize}
		\item Web Service
		\item Mobile Application
		\item Back-end
		\item Database
		\item Documentation
	\end{itemize}
	
	\subsection{Unit Testing}
	\begin{itemize}
		\item Web Service
		\item Mobile Application
		\item Back-end
		\item Database
		\item Documentation
	\end{itemize}
	
		
		\subsection{Integration Testing:}
		\begin{itemize}
			\item Describe integration strategy.
			\item Define individual steps and test description.
			\item Define program stubs.
			\item Latex document.
		\end{itemize}
	
	
	\subsection{Inspection:}
	\begin{itemize}
		\item Define functional rules.
		\item Describe assignment checklist.
		\item Latex document.
	\end{itemize}
	\newpage
	\section{Resources Allocation and Schedule}
	Here is presented the resource (team members) allocation with the relative deadline for each task. If a deadline is not present it means that that task is not assigned to the team member.
	\newline
	\newline
	\begin{tabularx}{\textwidth}{|l|X|c|c|}
		\hline
		\textbf{Task ID} &	\textbf{Description} & \textbf{Giacomello} & \textbf{Fontana} \\
		\hline
		1.1	 & Identify the project objectives. &  15/10/2015 & 15/10/2015 \\
		1.2 & Describe the structure and functions of the project.  & 15/10/2015  &  \\
		1.3 & 	Define the specific requirements. & 15/10/2015  &  \\
		1.4 & 	Identify the use cases and scenarios. &   & 15/10/2015 \\
		1.5 & 	Representing UML diagrams. & 30/10/2015 &  \\
		1.6 & 	Representing the model in Alloy. &   & 30/10/2015 \\
		1.7 & 	Documentation &  30/10/2015 & 30/10/2015 \\
		\hline
		\multicolumn{2}{c}{DESIGN DOCUMENT} &  & \\
		\hline
		2.1 & 	Describe the architecture of the system & 30/11/2015  & 30/11/2015 \\
		2.2 & 	Describe component view and deployment view & 30/11/2015  &  \\
		2.3 & 	Describe runtime view and component interfaces &   & 30/12/2015 \\
		2.4 & 	Describe algorithm design & 15/01/2016  &  \\ 
		2.5 & 	Define user interface design &   & 15/02/2016 \\
		2.6 & 	Documentation & 15/02/2016  & 15/02/2016 \\
		\hline
		\multicolumn{2}{c}{PLANNING} &  & 	\\
		\hline
		3.1 &	Define Function points & 30/02/2016 &  \\
		3.2 & 	Define Cocomo &   & 30/02/2016 \\
		3.3 & 	Describe task for the project and the schedule &  30/02/2016 & 30/02/2016 \\
		3.4 & 	Define the risks & 30/02/2016  & 30/02/2016 \\
		3.5 & 	Documentation & 30/02/2016 & 30/02/2016 \\
		\hline
			\end{tabularx}
			\begin{tabularx}{\textwidth}{|l|X|c|c|}
				\hline
		\multicolumn{2}{c}{DEVELOPMENT}	 &  & \\
		\hline
		4.1 & 	Web Service &   & 15/04/2016 \\
		4.2 & 	Mobile Application &   & 30/04/2016 \\
		4.3 & 	Backend & 30/04/2016  &  \\
		4.4 & 	Database & 15/04/2016  &  \\
		4.5 & 	Documentation & 30/04/2016  & 30/04/2016 \\
		\hline
		\multicolumn{2}{c}{UNIT TESTING} &  & \\
		\hline
		5.1 & 	Web Service & 30/04/2016  &  \\
		5.2 & 	Mobile Application & 30/04/2016  &  \\
		5.3 & 	Backend &   & 30/04/2016 \\
		5.4 & 	Database &   & 30/04/2016 \\
		5.5 & 	Documentation & 30/04/2016  & 30/04/2016 \\
		\hline
		\multicolumn{2}{c}{INTEGRATION TESTING}	 &  & \\
		\hline
		6.1 & 	Component Stubs & 30/06/2016  & 30/06/2016 \\
		6.2 & 	Phase 1 & 30/06/2016  &  \\
		6.3 & 	Phase 2 &   & 15/07/2016 \\
		6.4 & 	Phase 3 & 30/07/2016 &  \\
		6.5 & 	Phase 4 &   & 15/08/2016 \\
		6.6 & 	Documentation & 15/08/2016 & 15/08/2016 \\
		\hline
		\multicolumn{2}{c}{INSPECTION DOCUMENT} &  & \\
		\hline
		7.1 & 	Define functional rules & 15/09/2016 &  \\
		7.2 & 	Describe assignment check list &   & 15/09/2016 \\
		7.3 & 	Documentation & 15/09/2016 & 15/09/2016 \\
		\hline
	\end{tabularx}
	\newline
	\image{ganttdiagramm.png}
	\newpage
	
	\section{Risks and Management}
		\begin{tabularx}{\textwidth}{|X|l|X|}
			\hline
			\textbf{Risk} &	\textbf{Severity} & \textbf{Possible Resolution} \\
			\hline
	\end{tabularx}

	\section{Work Hours}
		\begin{itemize}
			\item \textbf{Edoardo Giacomello}: 
			\item \textbf{Mattia Fontana}: 
		\end{itemize}	
\end{document}